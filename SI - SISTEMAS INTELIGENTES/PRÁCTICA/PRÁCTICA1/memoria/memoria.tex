\documentclass[a4paper,12pt]{article}
\usepackage[utf8]{inputenc}
\usepackage{amsmath}
\usepackage{verbatim}
\usepackage{graphicx}
\usepackage{hyperref}

\title{Práctica 1: Búsqueda Heurística con A*}
\author{
	Universidad de Alicante\\
	Departamento de Ciencia de la Computación e Inteligencia Artificial\\
	Asignatura: Sistemas Inteligentes\\
	Curso 2024/2025\\
	\textbf{Nombre del Estudiante:} [Tu Nombre]\\
	\textbf{Fecha de Entrega:} 3 de noviembre de 2024
}
\date{}

\begin{document}
	
	\maketitle
	
	\newpage
	
	\tableofcontents
	
	\newpage
	
	\section{Introducción}
	Esta práctica se centra en la implementación y análisis del algoritmo A* para la búsqueda de caminos óptimos en un entorno de rejilla 2D. El objetivo principal es desarrollar un sistema que permita a un conejo encontrar la ruta más eficiente hacia una zanahoria, considerando diferentes tipos de terreno y obstáculos.
	
	El algoritmo A* es una técnica de búsqueda informada que combina las ventajas de la búsqueda en amplitud y la búsqueda guiada por heurística. Es ampliamente utilizado en videojuegos, sistemas de navegación y problemas de planificación de rutas debido a su eficiencia y garantía de encontrar el camino óptimo si existe.
	
	\section{Descripción del Problema}
	El problema consiste en un mapa representado por una rejilla de celdas cuadradas. Cada celda puede ser:
	\begin{itemize}
		\item Hierba (transitable, coste bajo)
		\item Agua (transitable, coste medio)
		\item Roca (transitable, coste alto)
		\item Obstáculo (no transitable)
	\end{itemize}
	
	El conejo (punto de inicio) debe encontrar el camino más corto hasta la zanahoria (punto objetivo), considerando los costes de movimiento y las calorías consumidas en cada tipo de terreno.
	
	Los movimientos permitidos son en 8 direcciones: horizontal, vertical y diagonal. Los costes asociados son:
	\begin{itemize}
		\item Movimiento horizontal o vertical: 1
		\item Movimiento diagonal: 1.5
	\end{itemize}
	
	Además, cada tipo de terreno tiene un coste en calorías:
	\begin{itemize}
		\item Hierba: 2 calorías
		\item Agua: 4 calorías
		\item Roca: 6 calorías
	\end{itemize}
	
	El objetivo es implementar el algoritmo A* para encontrar el camino óptimo, considerando tanto el coste del movimiento como el gasto calórico.
	
	\section{Implementación del Algoritmo A*}
	
	\subsection{Estructura de Datos}
	Para implementar el algoritmo A*, se han utilizado las siguientes estructuras de datos principales:
	\begin{enumerate}
		\item \textbf{PriorityQueue}: Para mantener la lista de nodos a explorar, ordenados por su valor f(n).
		\item Lista 2D de booleanos: Para llevar un registro de los nodos visitados.
		\item Lista 2D de flotantes: Para almacenar los costes g(n) de cada nodo.
		\item Lista 2D de cadenas: Para representar el camino final.
	\end{enumerate}
	
	Estas estructuras se inicializan en el método \texttt{buscar}:
	
	\begin{verbatim}
		pq = PriorityQueue()
		visited = [[False for _ in range(self.m)] for _ in range(self.n)]
		g_costs = [[float('inf') for _ in range(self.m)] for _ in range(self.n)]
		camino = [['.'] * self.m for _ in range(self.n)]
	\end{verbatim}
	
	\subsection{Funciones Principales}
	
	\subsubsection{init}
	\begin{verbatim}
		def __init__(self, mapa: Mapa, origen: Casilla, destino: Casilla):
		self.mapa = mapa
		self.origen = origen
		self.destino = destino
		self.n = mapa.getAlto()
		self.m = mapa.getAncho()
		self.dx = [-1, -1, 0, 1, 1, 1, 0, -1]
		self.dy = [0, 1, 1, 1, 0, -1, -1, -1]
		self.moves_map = [
		[8, 1, 2],
		[7, 0, 3],
		[6, 5, 4]
		]
		self.best_cost = float('inf')
		self.best_moves = []
	\end{verbatim}
	Esta función inicializa los atributos necesarios para el algoritmo A*. Se almacenan el mapa, las posiciones de origen y destino, y se definen las direcciones de movimiento posibles.
	
	\textbf{Complejidad temporal}: O(1), ya que todas las operaciones son de tiempo constante.
	
	\subsubsection{\texttt{calcular\_coste\_movimiento}}
	\begin{verbatim}
		def calcular_coste_movimiento(self, x1: int, y1: int, x2: int, y2: int) -> float:
		if x1 == x2 or y1 == y2:  # Movimiento horizontal o vertical
		return 1.0
		else:  # Movimiento diagonal
		return 1.5
	\end{verbatim}
	Esta función calcula el coste de movimiento entre dos celdas adyacentes. 
	
	\textbf{Complejidad temporal}: O(1), ya que solo realiza una comparación simple.
	
	\subsubsection{\texttt{generar\_hijos}}
	\begin{verbatim}
		def generar_hijos(self, node: Tuple[int, int, float, float, List[int]], pq: PriorityQueue, visited: List[List[bool]], g_costs: List[List[float]], heuristica):
		x, y, g, _, path = node
		
		for i in range(8):
		nx, ny = x + self.dx[i], y + self.dy[i]
		if 0 <= nx < self.n and 0 <= ny < self.m and self.mapa.getCelda(nx, ny) != 1:  # No es un muro
		new_g = g + self.calcular_coste_movimiento(x, y, nx, ny)
		if not visited[nx][ny] or new_g < g_costs[nx][ny]:
		h = heuristica(nx, ny)
		f = new_g + h
		new_path = path + [self.moves_map[self.dx[i] + 1][self.dy[i] + 1]]
		pq.put((f, (nx, ny, new_g, h, new_path)))
		g_costs[nx][ny] = new_g
	\end{verbatim}
	Esta función genera los nodos hijos del nodo actual, calculando sus costes y añadiéndolos a la cola de prioridad si son válidos y mejores que los existentes.
	
	\textbf{Complejidad temporal}: O(1) para cada hijo, ya que se generan 8 hijos como máximo.
	
	\subsubsection{\texttt{buscar}}
	\begin{verbatim}
		def buscar(self, heuristica=None) -> Tuple[float, List[List[str]]]:
		if heuristica is None:
		heuristica = self.heuristica_cero
		
		pq = PriorityQueue()
		visited = [[False for _ in range(self.m)] for _ in range(self.n)]
		g_costs = [[float('inf') for _ in range(self.m)] for _ in range(self.n)]
		camino = [['.'] * self.m for _ in range(self.n)]
		
		start_h = heuristica(self.origen.getFila(), self.origen.getCol())
		pq.put((start_h, (self.origen.getFila(), self.origen.getCol(), 0, start_h, [])))
		g_costs[self.origen.getFila()][self.origen.getCol()] = 0
		
		while not pq.empty():
		_, (x, y, g, h, path) = pq.get()
		
		if x == self.destino.getFila() and y == self.destino.getCol():
		self.best_cost = g
		self.best_moves = path
		break
		
		if visited[x][y]:
		continue
		
		visited[x][y] = True
		self.generar_hijos((x, y, g, h, path), pq, visited, g_costs, heuristica)
		
		if self.best_cost != float('inf'):
		x, y = self.origen.getFila(), self.origen.getCol()
		camino[x][y] = 'x'
		for move in self.best_moves:
		# ... (código para actualizar el camino)
		
		return self.best_cost, camino
	\end{verbatim}
	Esta es la función principal que implementa el algoritmo A*. Inicia la búsqueda desde el origen y continúa hasta encontrar el destino o agotar todos los nodos.
	
	\textbf{Complejidad temporal}: O(N log N), donde N es el número de nodos en el grafo. La operación más costosa es la inserción y extracción de la cola de prioridad, que tiene un coste de O(log N) por operación.
	
	\subsection{Heurísticas Implementadas}
	Se han implementado tres heurísticas diferentes:
	
	1. \textbf{Heurística de Manhattan}:
	\begin{verbatim}
		def heuristica_manhattan(self, x: int, y: int) -> float:
		return abs(x - self.destino.getFila()) + abs(y - self.destino.getCol())
	\end{verbatim}
	
	2. \textbf{Heurística Euclidea}:
	\begin{verbatim}
		def heuristica_euclidea(self, x: int, y: int) -> float:
		return math.sqrt((x - self.destino.getFila())**2 + (y - self.destino.getCol())**2)
	\end{verbatim}
	
	3. \textbf{Heurística Cero} (para búsqueda de coste uniforme):
	\begin{verbatim}
		def heuristica_cero(self, x: int, y: int) -> float:
		return 0
	\end{verbatim}
	
	Todas estas heurísticas tienen una complejidad temporal de O(1).
	
	\section{Análisis del Algoritmo}
	
	\subsection{Complejidad Temporal}
	La complejidad temporal del algoritmo A* depende principalmente de la heurística utilizada. En el peor caso, cuando la heurística no es informativa (como la heurística cero), el algoritmo puede explorar todos los nodos del grafo.
		\begin{itemize}
			\item Peor caso: \( O(b^{d}) \), donde \( b \) es el factor de ramificación y \( d \) es la profundidad de la solución.
			\item Mejor caso: \( O(d) \), cuando la heurística guía perfectamente al algoritmo hacia el objetivo.
		\end{itemize}

	En la práctica, con una buena heurística, el rendimiento suele estar entre estos dos extremos.
	
	\subsection{Complejidad Espacial}
	La complejidad espacial del A* es uno de sus principales inconvenientes, ya que necesita mantener todos los nodos generados en memoria.
	\begin{itemize}
		\item En el peor caso: \( O(b^d) \), donde \( b \) es el factor de ramificación y \( d \) es la profundidad de la solución.
	\end{itemize}

	Este es el motivo por el que se sugieren alternativas como A* con profundización iterativa o A* con memoria limitada para problemas con espacios de estados muy grandes.
	
	\section{Pruebas y Resultados}
	[En esta sección, deberías incluir los resultados de tus pruebas, comparando las diferentes heurísticas implementadas. Puedes usar tablas o gráficos para mostrar el número de nodos explorados y el tiempo de ejecución para diferentes mapas y heurísticas.]
	
	\section{Implementación de A*$\epsilon$}
	Aquí deberías explicar cómo has implementado el algoritmo A*$\epsilon$, sus diferencias con el A* estándar, y mostrar algunos resultados comparativos.
	
	\section{Conclusiones}
	Resume los principales hallazgos de tu implementación y análisis. Discute las ventajas y desventajas de las diferentes heurísticas y del algoritmo A*$\epsilon$ en comparación con el A* estándar.

	
	\section{Referencias}
	\begin{enumerate}
		\item Hart, P. E., Nilsson, N. J., \& Raphael, B. (1968). A Formal Basis for the Heuristic Determination of Minimum Cost Paths. IEEE Transactions on Systems Science and Cybernetics, 4(2), 100-107.
		\item Russell, S. J., \& Norvig, P. (2020). Artificial Intelligence: A Modern Approach (4th ed.). Pearson.
		\item Algorithmic Approaches to Playing Minesweeper. (n.d.). Retrieved [insert date] from \url{https://dash.harvard.edu/bitstream/handle/1/14398552/BECERRA-SENIORTHESIS-2015.pdf}
	\end{enumerate}
	
\end{document}
